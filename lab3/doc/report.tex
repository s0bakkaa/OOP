\documentclass[12pt]{article}

\usepackage{fullpage}
\usepackage{multicol,multirow}
\usepackage{tabularx}
\usepackage{ulem}
\usepackage[utf8]{inputenc}
\usepackage[russian]{babel}
\usepackage{minted}
\usepackage{color} %% это для отображения цвета в коде
\usepackage{listings} %% собственно, это и есть пакет listings
\lstset{ %
language=C++,                 % выбор языка для подсветки (здесь это С++)
basicstyle=\small\sffamily, % размер и начертание шрифта для подсветки кода
numbers=left,               % где поставить нумерацию строк (слева\справа)
%numberstyle=\tiny,           % размер шрифта для номеров строк
stepnumber=1,                   % размер шага между двумя номерами строк
numbersep=5pt,                % как далеко отстоят номера строк от подсвечиваемого кода
backgroundcolor=\color{white}, % цвет фона подсветки - используем \usepackage{color}
showspaces=false,            % показывать или нет пробелы специальными отступами
showstringspaces=false,      % показывать или нет пробелы в строках
showtabs=false,             % показывать или нет табуляцию в строках
frame=single,              % рисовать рамку вокруг кода
tabsize=2,                 % размер табуляции по умолчанию равен 2 пробелам
captionpos=t,              % позиция заголовка вверху [t] или внизу [b] 
breaklines=true,           % автоматически переносить строки (да\нет)
breakatwhitespace=false, % переносить строки только если есть пробел
escapeinside={\%*}{*)}   % если нужно добавить комментарии в коде
}


\begin{document}
\begin{titlepage}
\begin{center}
\textbf{МИНИСТЕРСТВО ОБРАЗОВАНИЯ И НАУКИ РОССИЙСКОЙ ФЕДЕРАЦИИ
\medskip
МОСКОВСКИЙ АВИАЦИОННЫЙ ИНСТИТУТ
(НАЦИОНАЛЬНЫЙ ИССЛЕДОВАТЕЛЬСКИЙ УНИВЕРСТИТЕТ)
\vfill\vfill
{\Huge ЛАБОРАТОРНАЯ РАБОТА №5} \\
по курсу объектно-ориентированное программирование
I семестр, 2021/22 уч. год}
\end{center}
\vfill

Студент \uline{\it {Шатунова Юлия Викторовна, группа М8О-208Б-20}\hfill}

Преподаватель \uline{\it {Дорохов Евгений Павлович}\hfill}

\vfill
\end{titlepage}

\subsection*{Условие}

Задание: \
Вариант 26: квадрат, очередь.\
Необходимо спроектировать и запрограммировать на языке C++ класс-контейнер первого
уровня, содержащий одну фигуру (колонка фигура 1), согласно вариантам задания. Классы
должны удовлетворять следующим правилам:
\begin{itemize}
	\item Требования к классу фигуры аналогичны требованиям из лабораторной работы №1;
	\item Требования к классу контейнера аналогичны требованиям из лабораторной работы №2;
	\item Класс-контейнер должен содержать объекты используя std::shared \_ ptr<...>
\end{itemize}


\subsection*{Описание программы}

Исходный код лежит в 10 файлах:
\begin{enumerate}
\item main.cpp: основная программа, взаимодействие с пользователем посредством комманд из меню

\item figure.h:    описание класса фигуры
\item tqueue\_item.h:    описание класса предмета очереди
\item point.h:     описание класса точки
\item tqueue.h:  описание класса очереди
\item square.h: описание класса квадрата, наследующегося от figures
\item point.cpp:     реализация класса точки
\item tqueue.cpp:  реализация класса очереди
\item square.cpp: реализация класса квадрата
\item tqueue\_item.cpp:    реализация класса предмета очереди

\end{enumerate}

\subsection*{Выводы}

Я научилась использовать "умные" указатели и применять их в построении и программировании классов.
\pagebreak

\subsection*{Исходный код}

{\Huge figure.h}
\inputminted{C++}{tqueue.h}
\pagebreak

{\Huge tqueue.h}
\inputminted{C++}{tqueue.h}
\pagebreak

{\Huge tqueue.cpp}
\inputminted{C++}{tqueue.cpp}
\pagebreak

{\Huge tqueue\_item.h}
\inputminted{C++}{tqueueitem.h}
\pagebreak

{\Huge tqueue\_item.cpp}
\inputminted{C++}{tqueueitem.cpp}
\pagebreak
    
{\Huge point.h}
\inputminted{C++}{point.h}
    \pagebreak

{\Huge point.cpp}
\inputminted{C++}{point.cpp}
    \pagebreak

{\Huge square.h}
\inputminted{C++}{square.h}
\pagebreak

{\Huge square.cpp}
\inputminted{C++}{square.cpp}
\pagebreak
    
{\Huge main.cpp}
\inputminted{C++}{main.cpp}
    \pagebreak
    
\end{document}